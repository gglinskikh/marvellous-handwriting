\documentclass[10pt]{book}

\usepackage[russian]{babel}
\usepackage[utf8x]{inputenc}

\usepackage{amsmath,amssymb,amsthm}
\usepackage{hyperref}
\usepackage{fancyvrb}

\author{Глинских Георгий}
\title{Конспект лекций по строковым алгоритмам}

\theoremstyle{plain}
\newtheorem{stm}{Утверждение}[section]
\newtheorem{definition}{Определение}[section]

\newcommand{\textm}{\texttt{text}}
\newcommand{\patm}{\texttt{pat}}
\newcommand{\algm}[1]{\operatorname{#1}}
\newcommand{\linem}{\noindent\rule{\textwidth}{1pt}}

% Можно потом попробовать p или t
\newenvironment{figurem}[1][]
  {\begin{figure}[p]
  \caption{#1}
  \centering
  }
  {
  \end{figure}
  }

\DefineVerbatimEnvironment%
  {verbm}{Verbatim}
  {gobble=-2,numbers=left,numbersep=2mm,
  frame=single,framerule=1pt}

\begin{document}

\maketitle

\chapter{Поиск подстроки в строке}

\section{Вводные замечания}

Строки -- конечные последовательности символов над множеством, называемым
алфавитом. Здесь и далее я буду полагать, что алфавит -- множество чисел
$\{1, \ldots, \sigma\}$ и что строки индексируются с $1$\footnote{Так
удобнее для математиков, но хуже для программистов.}:
$$w = w(1)w(2)\ldots{}w(n) = w(1,n).$$

Введем понятия периода и грани строки.
\begin{definition}
  \label{period_dfn}
  Периодом строки назовем число $p$, такое, что
  $$\forall i : s(i) = s(i + p).$$ 
\end{definition}

\begin{stm}
  Следующие определения равнозначны:
  \begin{enumerate}
    \item $p$ -- период строки $w$.
    \item $s(p + 1, n)$ и суффикс, и префикс $w$.
    \item Существует слова $a$, $b$: $b$ префикс $a$ и $w = a^k b$.
  \end{enumerate}
\end{stm}
\begin{proof}

$1 \Rightarrow 2$. Посимвольно сравнивая, $s(1, n-p) = s(p + 1, n)$.

$2 \Rightarrow 3$. Положим $a = s(1, p)$. Тогда видно, что
$s(p + 1,2p) = s(1,p) = a$,
Аналогично $$\forall l : s((l-1)p + 1, lp) = a.$$
Остается взять $b = s(kp + 1, n)$, где $k$ -- наибольшее из чисел $l$.

$3 \Rightarrow 1$. Очевидно.

\end{proof}

\begin{definition}
  \label{border_dfn}
  Гранью строки назовем суффикс, одновременно являющийся префиксом. 
\end{definition}

\begin{stm}
  Число $0 < p \le n$ -- период $s$ согда $s(1, n-p)$ -- грань $s$.
\end{stm}
\begin{proof}
  Заметим, что $s(1, n-p) = s(p + 1, n)$ по условию, и тогда достаточно
  воспользоваться пунктом 2 предыдущего утверждения.
\end{proof}

\begin{quote}
  Важно мыслить объекты, которые обрабатываются алгоритмами, очень (очень!)
  большой размерности. Тогда будет появляться нужная интуиция.
\end{quote}

\section{Алгоритм Крошмора-Перрена}

При помощи периода можно улучшить обычный алгоритм поиска подстроки в строке.
Далее будем называть его Ord, его сложность
$T(\algm{Ord}) = O(|\textm||\patm|)$.

\begin{quote}
  Инвариант цикла в Ord -- вхождения с началом в $\textm(1, i)$ рассмотрены.
\end{quote}

\begin{figurem}[Обычный алгоритм поиска подстроки в строке (Ord)]
\begin{verbm}
  i = 1
  пока i + |pat| <= |text|:  # O(|text|)
    если text(i, i + |pat|) = pat(1, |pat|):  # O(|pat|)
      верни i
    i += 1
\end{verbm}
\end{figurem}

Пусть теперь известен $p$ -- минимальный период строки $\patm$. Тогда этот
алгоритм можно улучшить для нахождения серии вхождений
(вхождения, расположенные достаточно близко друг к другу). Заметим, что
вхождения находятся на расстоянии не меньше $p$ (иначе противоречие с 
минимальностью периода), следовательно, при
нахождениии вхождения, можно не смотреть на следущие $p$ возможных позиций.
Получим алгоритм Per.
\begin{quote}
  Инвариант цикла в Per -- для нахождения вхождения 
  остается рассмотреть $|\patm| - b + 1$ позиций. Иными словами,
  $\patm(1,b) = \textm(i, i+b)$.
\end{quote}

\begin{figurem}[Улучшенный алгоритм поиска подстроки в строке (Per)]
\begin{verbm}
  i = 1, b = 1
  пока i + |pat| <= |text|:  # O(|text| / p)
    c = max { c : pat(b, c) = text(i + b, i + c) }  # O(|pat|)
    если с < |pat|:
      i += 1, b = 1
    иначе:  # c = |pat|
      верни i
      i += p, b = |pat| - p + 1
\end{verbm}
\end{figurem}

Теперь рассмотрим наконец алгоритм Крошмора-Перрена. Его будем обозначать TW:
в англоязычной литературе его называют two-way алгоритмом. Он быстрый:
$T(\algm{TW}) = O(|\textm| + |\patm|), S(\algm{TW}) = O(1)$.

\begin{definition}
  Локальным периодом разложения $w = ab$ назовем число $q$ такое, что
  $$\forall i \in \{|a| - q, \ldots, |a|\}: \quad s(i) = s(i + q).$$
\end{definition}

Заметим, что число $p$ -- период $w$ будет локальным периодом для любого
разложения.

\begin{definition}
  Критическим назовем разложение $w = ab$, для которого минимальный локальный
  период совпадает с периодом строки $w$.
\end{definition}

\begin{stm}
  \label{crit_stm}
  У любой строки есть критическое разложение $w = ab$: $|a| = u < p$, где $p$
  -- период $w$. 
\end{stm}
\begin{proof}

Заметим сначала, что любой локальный период $q$ для критического разложения 
$(a,b)$ с условием $q < |b|$ делится на $p$. Из критичности, $q \ge p$. Тогда 
$(xy)^kx$ с $|xy| = p, |x| < p$ будет префиксом $b$. Но тогда $x$ будет 
суффиксом $a$ и префиксом $b$. Пришли к противоречию: удалось найти локальный
период $|x| < p \le q$.
  
\end{proof}

Рассмотрим алгоритм TW1. Идейно он работает так: зафиксируем критическое 
разложение $\patm = ab$ как в утверждении \ref{crit_stm} и
сначала пытаемся найти вхождение $b$ слева направо,
после чего проверим справа налево, будет ли это вхождение иметь вид
$ab = \patm$.

В начале каждой итерации главного цикла соблюдается инвариант
$\patm(1,b) = \textm(i,i+b)$ (аналогичено алгоритму Per).

\begin{figurem}[Алгоритм TW1]
\begin{verbm}
  # вход: text: str, pat: str, u: num, p: num 
  # ограничения: u < p, u критическая для pat, p - период pat
  i = 1, b = 1
  пока i + |par| <= |text|:
    c = max(u, b)
    d = max { d : pat(c, d) = text(i + c, i + d) }
    если d < |pat|:
      i += d - u
      b = 1
    иначе:  # d = |pat|
      c = u - 1
      d = min { d >= b : pat(d, c) = text(i + d, i + c) }
      если d < b:
        верни i
      i += p
      b = |pat| - p + 1
\end{verbm}
\end{figurem}

\begin{stm}
  \label{TW1_corr}
  Алгоритм TW1 находит все вхождения $\patm$ в $\textm$ со сложностью
  $$T(\algm{TW1}) = O(|\textm|), \quad S(\algm{TW1}) = O(1).$$
\end{stm}
\begin{proof}
  Допустим, $i$ пробегает позиции $\{i_k\}$, и в позиции $i_k < i' < i_{k+1}$
  было вхождение, которое мы не выведем. Одно из двух: мы при сканировании
  перескочили этот индекс или мы не распознали вхождение по этому индексу.

  Если мы перешли к $i_{k + 1}$ через строку 8, то мы не можем пропустить
  вхождение в силу критичности разложения и $|a| < p$. Формально,
  $i' - i_k < d - u$, тогда по утверждению \ref{crit_stm}, $i - i_k = Ap$.
  Следовательно, 
  $\textm(i + c) \ne \patm(c) = \patm(c - Ap) = \textm(i' - c - i')$.
  Противоречие.

  Если мы перешли к $i_{k+1}$ через строку 15, то мы не можем пропустить в
  силу того, что $p$ -- наименьший локальный период.

  Итак, алгоритм находит все вхождения.
  
  Мы используем лишь несколько переменных, откуда $S(\algm{TW1}) = O(1)$.
  Для оценки времени работы $T(\algm{TW1})$, заметим, что в 6 и 12 строке мы
  не касаемся символа дважды.
\end{proof}

Далее необходимо найти разложение и период. Идейно надо сделать пользоваться
алгоритмом $\algm{TW2}$.

\begin{figurem}[Алгоритм TW2]
\begin{verbm}
  # вход: крит. pat = ab, |a| < p. p' - период b.
  если a - суффикс b(1, p'):
    p = p'
    алгоритм TW1
  иначе:
    q = max {|a|, |b|} + 1
    алгоритм TW3.
\end{verbm}
\end{figurem}

\begin{figurem}[Алгоритм TW3]
\begin{verbm}
  # вход: text: str, pat: str, u: num, q: num 
  # ограничения: u < p, u критическая для pat
  i = 1
  пока i + |par| <= |text|:
    c = u
    d = max { d : pat(c, d) = text(i + c, i + d) }
    если d < |pat|:
      i += d - u
    иначе:  # d = |pat|
      c = u - 1
      d = min { d >= 0 : pat(d, c) = text(i + d, i + c) }
      если d = 0:
        верни i
      i += q
\end{verbm}
\end{figurem}

\begin{stm}
  Алгоритм TW3 находит все вхождения $\patm$ в $\textm$ со сложностью
  $$T(\algm{TW1}) = O(|\textm|), \quad S(\algm{TW1}) = O(1).$$
\end{stm}
\begin{proof}
  Ясно, что $p' < p$. Причем $p'$ -- период $\patm$ согда верно условие 
  в строке 2. Остальное доказательство повторяет доказательство утверждения
  \ref{TW1_corr}.
\end{proof}

\begin{stm}
  В алгоритме $\algm{TW2}$ $$q = max \{|a|, |b|\} + 1 < p.$$
\end{stm}
\begin{proof}
  Известно, что $|a| < p$. Достаточно показать, что $|b| < p$. Это так в силу
  того, что иначе если $p' | p$, то $p'$ -- локальный период. Иначе найдется
  период еще меньше, чем $p'$.
\end{proof}

Займемся необходимой предобработкой. Обозначим через $<$ лексикографический
порядок символов, а через $\lessdot$ -- обратный к лексикографическому.
Продолжим их на строки.

\begin{stm}[Волшебное разложение]
  Пусть $\patm = ab = cd$ где $b$ и $d$ -- лексикографически максимальные по
  $<$ и $\lessdot$ суффиксы. Если $|b| < |d|$, то $\patm = ab$ критичное.
  Иначе $\patm = cd$ критичное.
\end{stm}
\begin{proof}
  Если строка из одинаковых символов, то очевидно. Иначе не умаляя общности,
  пусть $|b| < |d|$.
  
  Заметим, что локальный период разложение $\patm = ab$ больше $|a|$. Ведь
  в противном случае можно прийти к противоречию с максимальностью $b$.

  Осталось показать, что $q$ -- минимальный период для разложения $\patm = ab$
  будет также периодом $\patm$. Используем такой факт: для любых строк $x, y$:
  $x < y$ и $x \lessdot y$, $x$ будет префиксом $y$. Используем его для строк
  $\patm(|c|+q,\cdot)$ и $\patm(|c|,\cdot)$.

  Очевидно, $\patm(|c|+q,\cdot) \lessdot \patm(|c|,\cdot)$ в силу выбора
  $c$. По порядку $<$ первые символы совпадают, а для остальных порядок, как
  между $\patm(|a|, \cdot) = b$ и $\patm(|a| + q, \cdot)$. Получили
  требуемое условие.
\end{proof}

Осталось найти максимальные суффиксы по заданным порядкам. Для этого используем
модификацию алгоритма Дюваля ($\algm{TW4}$). Идейно: читаем строку слева направа
и для каждого префикса находим максимальный суффикс и его период.

\begin{stm}[Дюваль]
  Пусть $\patm(i,k)$ -- максимальный суффикс строки $\patm(1,k)$.
  \begin{enumerate}
    \item Если $\patm((k+1) - p) = \patm(k+1)$, то $\patm(i,k+1)$ новый суффикс,
      его период $p$.
    \item Если $\patm((k+1) - p) > \patm(k+1)$, то $\patm(i,k+1)$ новый суффикс,
      его период $k - i$ (тривиальный).
    \item Если $\patm((k+1) - p) < \patm(k+1)$, то максимальный суффикс не может
      начинаться в позициях $[0, i + \{d: d|p, d > r \}]$.
  \end{enumerate}
\end{stm}
\begin{proof}
  Случай 1 очевиден. В остальных случаях от противного, и рассматриваем грани.
  Там можно вывести противоречие с максимальностью суффикса на предыдущем шаге.
\end{proof}

\begin{figurem}[Алгоритм TW4 (Дюваля)]
\begin{verbm}
  # вход: pat: str
  p = <j - i>
  i = 1, j = 2
  пока j <= |pat|:
    d = max { j + d <= |pat| : pat(i, i + d) = pat(j, j + d) }
    если j + d > |pat|: выйди из цикла
    если pat(i + d) > pat(j + d): j += d + 1
    иначе: i += (d / p + 1), j = i + 1
  верни pat(i, j - 1), p
\end{verbm}
\end{figurem}

Для оценки сложности заметим, что значение $2i + j$ за $k$ итераций 
увеличивается не меньше, чем на $k$.

\chapter{Поиск нескольких подстрок в строке}

\section{Алгоритм Ахо-Корасик}

Назовем множество строк $D$, которое будем называть словарем. Задача состоит в
нахождении всех вхождений всех слов из $D$ в строку $\textm$.

Задача решается в два этапа: составление вспомогательной структуры данных $P$
и обработка строки $\textm$. Есть наивный алгоритм $\algm{TR}$: в нем в
качестве вспомогательной структуры берется бор, а вершины - структуры:
\begin{verbm}
  root : V = < корень бора >
  
  v : V = {
    .repr = < строка на пути root - v >  # формально
    .term = < .repr является словом словаря D >
    .next = < (-): char -> V : .next(c).repr = .repr + c >
  }
\end{verbm}

\begin{figurem}[Алгоритм TR]
\begin{verbm}
  v = root
  для i = 1; i < |text| и v != nil; i += 1:
    если v.term: нашли слово v.repr в позиции i
    v = v.next(text(i))
\end{verbm}
\end{figurem}

Сложность алгоритма $\algm{TR}$ будет зависить от реализации $(-).next$. Если
использовать отображение или отсортированный массив пар, то время доступа
$O(\log \sigma)$. Если lookup-таблицу или массив, то $O(1)$. Для удобства будем
представлять вершины числами $\{1, 2, \ldots\}$ и что $.next$ реализован
lookup-таблицей.

Улучшим алгоритм $TR$: давайте добавим поле $v.link$, в котором
будем указываем на вершину для которой $u.repr$ будет наидлиннейшим суффиксом из
$v.repr$. $v.report$ будет указывать на вершину, которая при спуске по $v.link$
будет удовлетворять $.term = true$. Тогда получаем алгоритм Ахо-Корасик
$\algm{AHK}$

\begin{figurem}[Алгоритм AHK]
\begin{verbm}
  v = root
  для i = 1; i < |text| и v != nil; i += 1:
    пока v != root и v.next(text(i)) = nil: v = v.link
    v = v.next(text(i))
    если v = nil:
      v = root
    для u = v; u != root; u = u.report:
      если u.term:
        нашли слово u.repr в позиции i - |u.repr|
\end{verbm}
\end{figurem}

\begin{stm}
  Алгоритм $\algm{AHK}$ работает корректно и
  $$T(\algm{AHK}) = O(|\textm| \log \sigma + |D|)$$
\end{stm}
\begin{proof}
  Каждая итерация первого цикла увеличивает $j = i - |v.repr| \le i$. Тогда он
  отработает за $O(|\textm| \log \sigma)$. Второй цикл каждый раз вызывается
  для разных слов, так что он будет вызван $O(|D|)$ раз.

  Корректность следует из того, что $\algm{AHK}$ -- просто модицикация алгоритма
  $\algm{TR}$.
\end{proof}

Осталось научиться находить поле $v.link$. Для проверки максимального символа мы
можем отрезать по одному символу $v.repr$ и смотреть, в $u.repr$ какой вершины
мы перешли. Получим алгоритм $\algm{AHK1}$.

\begin{stm}
  $T(\algm{AHK1}) = O(\log \sigma \sum_{d \in D} |d|)$.
\end{stm}
\begin{proof}
  Это верно из того, что на пути $root-<v:v.repr = d>$ суммарное время работы для
  всех вершин будет $O(|d|)$:
  $$\sum n_i \le \sum |v_i.link.repr| - |v_d.link| + 1 = |v_d.link.repr| + |d|$$
\end{proof}

\begin{figurem}[Алгоритм AHK1]
\begin{verbm}
  qu(1) = root.link = root.report = root
  k = 2;
  для i = 1; i < k; i += 1:
    для v,c из { (v,c) qu(i).next(c) = v }:
      qu[k++] = v
      v.link = nil
      для p = qu[i]; p.report != nil и v.link = nil; p = p.link:
        v.link = p.link.next(c)
      если v.link = nil:
        v.link = root
      v.report = v.link
      пока !v.report.term:
        v.report = v.link.report
\end{verbm}
\end{figurem}

Если в словаре только одно слово, то получаем алгорит Кнута-Морриса-Пратта.
Тогда вместо дерева достаточно использовать массив значений. Вариант выше
занимает $S(\algm{AHK}) = O(|V|)$ памяти, а автоматный вариант $O(\sigma|V|)$
памяти.

\chapter{Строковые индексы}

\section{Суффиксный массив}

По строке $\textm$ необходимо построить структуру данных, с помощью которой
можно находить вхождения строки $\patm$ в строку $\textm$.

\begin{definition}
  Cуффиксный массив $SA$ содержит перестановку чисел $\{1,2,\ldots,|\textm|\}$,
  причем $\textm(SA(i), \cdot) < \textm(SA(j), \cdot)$ для $i < j$.
\end{definition}

\subsection{Алгоритм Карккайнена-Сандерса}

Допустим, что алфавит представлен числами $\{1,2,\ldots,|\textm|\}$.
Идейно можно бить строку на фрагменты длины $2, 4, 8, \ldots$ и сортировать
их слиянием: тогда сложность
$T(n) = O(n) + T({n \over 2}) = cn \sum_k {1 \over 2^k} = O(n)$.

Идейно: рекурсивно отсортируем все суфииксы: сначала на позициях не кратных
трем, затем оставшиеся и выполним слияние.

Пусть на каком-то этапе
отсортированными оказываются строки $t_1, t_2$ (порязрядной сортировкой).
И в строке $t_1$ все тройки меньше, чем в строке $t_2$. Пронумеруем тройки
числами $\{1, {2 \over 3}|\textm|\}$. Сформируем строку $t_1 \$ t_2$ с
бесконечно малым символом \$. Рассмотрим суффиксный массив $SA_{12}$ для этой
строки. Его занесем в $ISA_{12}(SA(x)) = x$.

Для сортировки оставшихся троек: если первые символы различны, то порядки
определяются непосредственно, иначе -- по построенным порядкам. Тогда,
получается, достаточно отсортировать пары $(\textm(3k), ISA_{12}(k))$.

Для слияния так же переиспользуем порядок, зафиксированный в $ISA_{12}$.

\section{Cуффиксное дерево}

\begin{definition}
  Массив $LCP$, содержит на позиции $i$ наибольший общего префикс строк
  $\textm(SA(i), \cdot)$ и $\textm(SA(i + 1), \cdot)$
\end{definition}

Если построить структуру,
вычисляющую $\min \{ LCP(x) : i < x < j \}$ за $O(1)$, то можно построить
наибольший общий префикс любой пары суффиксов за $O(1)$. $LCP$ можно построить
за $O(|\textm|)$ используя $SA, ISA$ по алгоритму Касаи и др.

\begin{definition}
  Суффиксный бор -- бор, в котором пути отмеченны всеми подстроками строки
  $\textm$. Терминальными вершинами считаются суффиксы.
\end{definition}

\begin{definition}
  Суффиксным деревом (сжатым суффиксным бором) назовем суффиксный бор без
  вершин с одним сыном. При удалении вершин соответствующие метки ребер
  склеиваются.
\end{definition}

Метками ребер оказываются подстроки. Будем хранить указатели на них. Тогда
вершины являются структурами:
\begin{verbm}
  root : V = < корень бора >
  
  v : V = {
    .repr = < строка на пути root - v >  # формально
    .term = < .repr является суффиксом text >
    .next = < по символу переходим в вершину,
              на ребре до которой данный символ первый>
    .par = < родитель >
    .beg, .end = < text(.beg, .end) = str(.par, v) >
  }
\end{verbm}

\begin{quote}
  Из существования полей $.beg, .end$, очевидно, у суффиксного дерева
  $|V| \le 2|\textm|$.
\end{quote}

Суффиксное дерево можно построить по $SA$ и $LCP$. Для этого надо вставлять
суффиксы в порядке $(\textm(SA(1), \cdot), \textm(SA(2), \cdot), \ldots)$.
При вставке очередного суффикса $\textm(SA(1), \cdot)$, при вставке поднимаемся
снизу этого пути, создавая вершину на нужном месте.

\end{document}